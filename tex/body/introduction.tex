% !TEX TS-program = pdflatex
% !TEX encoding = UTF-8 Unicode
% !TEX root = ../main.tex
% !TEX spellcheck = en-US
% ****************************************************************************************
% File: introduction.tex
% Author: Jakob Spindler
% Date: 2024-10-16
% ****************************************************************************************
\chapter{Introduction}
\label{chapter:introduction}

A discrete \gls{acr:pid} controller for a \qty{12}{\volt} to \qty{5}{\volt} Buck-Converter is to be designed and tested using PLECS \autocite{PLECSPlexim}.

The self-imposed charcteristics of the buck converter and the chosen passive components according to \autocite{BuckConverter} are given in \autoref{tab:converter_characteristics} and \autoref{tab:components} respectively.

\begin{table}[htbp]
    \centering
    \begin{minipage}{0.45\textwidth}
        \centering
        \begin{tabular}{c|c}
            Parameter & Value \\ \hline
            $V_{\text{in,nominal}}$ & \qty{12}{\volt} \\ 
            $V_{\text{in,min}}$ & \qty{8}{\volt} \\
            $V_{\text{in,max}}$ & \qty{16}{\volt} \\  
            $V_{\text{out}}$ & \qty{5}{\volt} \\ 
            $I_{\text{out}}$ & \qty{1}{\ampere} \\
            $\Delta I_{\text{L}}$ at $V_{\text{in,max}}$ & \qty{400}{\milli\ampere} \\
            $f_{\text{sw}}$ & \qty{50}{\kilo\hertz}
        \end{tabular}
        \caption{Buck-Converter Characteristics}
        \label{tab:converter_characteristics}
    \end{minipage}\hfill
    \begin{minipage}{0.45\textwidth}
        \centering
        \begin{tabular}{c|c}
            Component & Value \\ \hline
            $L$ & \qty{200}{\micro\henry} \\ 
            $C$ & \qty{50}{\micro\farad} \\ 
            $R$ & \qty{5}{\ohm} \\ 
             \\
        \end{tabular}
        \caption{Passive components \\ of the Buck-Converter}
        \label{tab:components}
    \end{minipage}
\end{table}

A continous controller can be designed according to \autocite{samosirSimpleFormulaDesigning2023} -- for simplicity, the parameters were kept the same for the discrete controller

\begin{table}[htbp]
    \centering
    \begin{tabular}{c|c}
        Parameter & Value \\ \hline
        $K_D$ & $50 LC$ \\ 
        $K_P$ & $50 \frac{L}{R}$ \\ 
        $K_I$ & $50$ \\ 
    \end{tabular}
    \caption{PID Controller Parameters}
    \label{tab:pid_parameters}
\end{table}


% EOF